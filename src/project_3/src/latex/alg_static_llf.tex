\begin{frame}
    \frametitle{ Least Laxity First (\textbf{LLF}) }

    \begin{exampleblock}{Descripción}
        Es un algoritmo para sistemas operativos de tiempo real con asignación de prioridades dinámicas basadas en la laxitud de la tarea.
    \end{exampleblock}

    \begin{itemize}
        \item Las prioridades son asignadas de acuerdo a la laxitud, que cambia en cada instante de tiempo.
        \item La tarea con la menor laxitud en un instante determinado va a tener la prioridad más alta que las restantes tareas en ese instante.
        \item La laxitud se describe como:
        \begin{equation}
            L_i = D_i - (t_i + C^r_i)
        \end{equation}
        
        \begin{itemize}
            \item $D_i$: es el siguiente \textit{deadline} de la tarea en el instante $t_i$.
            \item $t_i$: el instante de tiempo actual.
            \item $C^r_i$: el tiempo de computación restante de la tarea en el instante $t_i$.
        \end{itemize}
    \end{itemize}
\end{frame}